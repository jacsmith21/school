\documentclass[11pt]{article}
\usepackage[utf8]{inputenc}

\title{Generative Adversarial Networks}
\author{Jacob Smith}
\date{December 12, 2017}

\newcommand{\bb}{\textbf}
\newcommand{\ii}{\textit}

\usepackage{natbib}
\usepackage{graphicx}
\usepackage{bm}
\usepackage{algorithm2e}
\usepackage{parskip}
\usepackage{geometry}
\usepackage{enumitem}

\graphicspath{{./img/}}

\begin{document}
\maketitle

\section{Introduction}
This report details the experimental study conducted for the CS6735 programming project. Six machine learning algorithms were implemented, including both regular and ensemble learning algorithms. Each was evaluated on five datasets from the UCI Machine Learning data repository using 10 times 5-fold cross validation. The technical details of the implementation will be described in this report and the results will be presented and analyzed.

\section{Objectives}
\begin{enumerate}
  \item Implement the following machine learning algorithms:
  \begin{enumerate}
    \item ID3
    \item Adaboost on ID3
    \item Random Forest
    \item Naïve Bayes
    \item Adaboost on Naïve Bayes
    \item K-nearest neighbors
  \end{enumerate}
  \item Evaluate the algorithms on the following datasets:
  \begin{enumerate}
    \item Brest Cancer Data
    \item Car Data
    \item Ecoli Data
    \item Mushroom Data
  \end{enumerate}
  \item Create report of the experimental study including:
  \begin{enumerate}
    \item Description of the learning algorithms you implement.
    \item Description of the datasets you use (number of examples, number of attribute, number of classes, type of attributes, etc.).
    \item Technical details of your implementation: pre-processing of data sets (discretization, etc.), parameter setting, etc.
    \item Design of your programming implementation (data structures, overall program structure).
    \item Report and analysis of your experimental results using using 10 times 5-fold cross-validation (include mean, standard deviation of accuracies).
    \item Comparison and discussion of the algorithms with respect to the experimental results.
  \end{enumerate}
\end{enumerate}

\section{Algorithms}
For this study, six different algorithms; however, only the five unique algorithms are described in this section. Both the ID3 decision tree and Naïve Bayes algorithms were applied to the adaboost ensemble algorithm as weak learners.

\subsection{ID3}
The Iterative Dichotomiser 3 (ID3) algorithm is a decision tree algorithm used to efficiently develop decision trees \cite{Quinlan:1986:IDT:637962.637969}. This algorithm was developed for situations where there are large amount of attributes and training sets which contain many attributes, but there was the need to be able to efficiently develop relatively well performing decision trees.

The algorithm as described by Mitchell \cite{Mitchell:1997:ML:541177} was implmented for this assignment. The operations begin at the root node with a set of catigorical, labled data $S$. The attribute $A$ which introduces the largest information gain is chosen to split on. This is an exhaustive process which checks all remaining attributes. This process imediatly stops if there are no possible subsets (identical samples have same class label) or the split is not statistically significant (positive information gain). If these stopping conditions are not met, the data $S$ is split using attribute $A$ into subsets ${S_1,...,S_n}$ where $n$ is the amount of different categories of attribute $A$. A new node is then created for each subset. If the entropy of a new node is $0$ or the sample count is $1$, then the node is turned into a leaf and the splitting operation ends. Furthermore, two additional early stopping conditions are implemented. Max tree level and minimum number of samples attributes are initially set at the start of training.

After the initial ID3 algorithm was implemented, furthure enhancements were made to allow for continuous data to be used. This enhancement was taken from the C4.5  as developed by Ross Quinlan \citep{c45algorithm}. Given a continuous attribute $A$, we temporarily sort the data $S$. After the data is sorted, we iteratively determine a threshhold to split the data as the maximize the information gain. To test potential threshholds, we find two sequential numbers which are not of the same category and split at the midway point of the two values. This enhancements saves an a large amount of time for each continuous attribute found within the datasets as descritization does not need to occur. Furthermore, this algorithm chooses the optimal threshhold to split the data which improves classification accuracy.

\subsection{Naïve Bayes}
The Naïve Bayes algorithm is implemented as described in \cite{Mitchell:1997:ML:541177}. This algorithm is named due its assumption of conditional independence amongst each attribute:

$$P(a_1,...,a_n|v_j) = \prod_{i=1}^{n}P(a_i|v_j)$$

and its application of Bayes thereon:

$$P(A|B) = \frac{P(B|A)P(A)}{P(B)}$$

Naïve Bayes is a practical learning algorithm, receives relatively good results given its preference bias and is easy to implement. This algorithm naturally handles both categorical and continuous data (given a prior is assumed). For each continuous attribute, a Gaussian distribution was assumed. The final algorithm implemented was:

$$\hat{y}= \mathrm{argmax}_{k\in\{1,..K\}} P(v_k)\prod_{i=1}^mp\left(a_i|v_k\right)\prod_{j=m}^nP\left(a_j|v_k\right)$$

where $P$ is the probability, $p$ is the gaussian probability density function and $K$ is the number of distinct targets. In this equation, the first m attributes are continuous whereas the later are catigorical. During training, the data $S$ is split by lables into subsets ${S_1,...,S_n}$ where $n$ is the number of different classes. For each subset, we estimate the class probability $P(v_k)$. Subsequently, for each attribute value $a_i$ of each attribute $a$, we estimate $P(a_i|v_k)$ or $p$. We are not able to immediately calculate $p(a_i|v_k)$ as attribute $a$ is a continous attribute.

\subsection{Adaboost} \label{adaboost}
Adaboost is a boosting algorithm that was first introduced in 1995 by Freunb Schapire. This type of ensemble learning uses a set of weak learners to reduce bias and become a strong learner. It solves many of the difficulties associated with previous boosting algorithms \cite{Schapire:1999:BIB:1624312.1624417}.

The primary part of the algorithm is to maintain a weight value $w$ for every sample within the data $S$. This value increases when the sample is misclassified and decreases when the sample is correctly classified. The target values are assumed be to in the set ${+1, -1}$. The Adaboost algorithm is easily implemented on weak learners which are parameterized by sample weights. Neither Naïve Bayes or ID3 natively implement weighted training. Therefore, an alternative implementation was used where a subset of the training data is selected with replacement based on the weights. For example, if the weight vector for two samples is ${1, 2}$, the second sample is twice as likely to be picked each time.

Initially, the weights are set to $\frac{1}{|S|}$. For a predefined number of iterations, a weak learner is trained on the a subset of the traning data. Predictions are made for each training sample and the error is calculated. Using the accuracy, a value $\alpha$ is calculated which represents the accuracy of the weak learner. This $\alpha$ value is used to update the weights and the weak learner, $\alpha$ pair is added to the list of previous weak learners. To make classification, there is a weighted vote between using the weak learners and their associated $\alpha$ values and the sign is used to make the final classification.

To implement multiclass adaboost, $K$ adaboost models are trained where $K$ is the total number of distant classes. For each adaboost model, one target is selected as the $+1$ adaboost target and the other become the $-1$ target. The target value associated with the most positive prediction becomes the predicted class.

\subsubsection{Adaboost on Naïve Bayes}
Naïve Bayes is an inherently weak learner due to its assumption of conditional independence amongst attributes. This fact introduces a large amount of preference bias which makes a Naïve Bayes and ideal candidate for the Adaboost algorithm.

\subsubsection{Adaboost on ID3}
The ID3 algorithm does not introduce a large amount of preference bias and often overfitting if allowed to train to completion. Therefore, to implement Adaboost on ID3, the depth of the trees were limited to one level. This type of tree is known as a \ii{decision stump}.

\subsection{Random Forest}
Random forest algorithm is a bagging algorithm which uses an ensemble of strong learners to reduce overfitting to the training set. This method contains many similarities to the adaboost algorithm as described in section \ref{adaboost}; however, no weights are used for random forests. For a predefined number of iterations, a subset of the of the training data is selected with replacement. The new training set is then used to train a decision tree to completion and added to tree is added to a list. To make a classification, each tree in the forest votes and the most common target value becomes the predicted class. As an extension of the base random forest algorithm, weights may be given to each decision tree based on their classification error during training; however, this extension was not added for this implementation.

\subsection{K-Nearest Neighbors}
The K-Nearest Neighbor algorithm is an instance based learning algorithm which makes predictions by finding the $k$ nearest neighbors of a sample $x$ and using the most common class as the predicted class for that sample. This algorithm can be used with both categorical and continuous data. For this implementation, Euclidian distance was used for continuous variables and Hamming distance was used for categorical data.

\section{Data}
Each algorithm was implemented on five different datasets from the UCI Machine Learning Repository.

\subsection{Breast Cancer Data}
\begin{itemize}[leftmargin=*]
  \item[] \bb{Title}: Wisconsin Diagnostic Breast Cancer
  \item[] \bb{Date}: November 1995
  \item[] \bb{Number of Samples}: 569
  \item[] \bb{Number of Attributes}: 10
  \item[] \bb{Attribute Information}:
  \item[]
  \begin{tabular}{l c c c }
    \bb{Description}             & \bb{Values} & \bb{Type} \\
    Sample Code Number           & id number   & Discrete  \\
    Clump Thickness              & 1 - 10      & Discrete  \\
    Uniformity of Cell Size      & 1 - 10      & Discrete  \\
    Uniformity of Cell Shape     & 1 - 10      & Discrete  \\
    Marginal Adhesion            & 1 - 10      & Discrete  \\
    Single Epithelial Cell Size  & 1 - 10      & Discrete  \\
    Bare Nuclei                  & 1 - 10      & Discrete  \\
    Bland Chromatin              & 1 - 10      & Discrete  \\
    Normal Nucleoli              & 1 - 10      & Discrete  \\
    Mitoses                      & 1 - 10      & Discrete
  \end{tabular}
  \item[] \bb{Class Target}: Neoplasm
  \item[] \bb{Class Information}:
  \item[]
  \begin{tabular}{l c c c c }
    \bb{Description} & \bb{Value} & \bb{Count} & \bb{Percentage} \\
    Benign           & 2          & 357        & 65.5\%          \\
    Malignant        & 4          & 212        & 34.5\%
  \end{tabular}
  \item[] \bb{Missing Attributes}: 16
\end{itemize}

\subsection{Car Data}
\begin{itemize}[leftmargin=*]
  \item[] \bb{Title}: Car Evaluation Database
  \item[] \bb{Date}: June 1997
  \item[] \bb{Number of Samples}: 1728
  \item[] \bb{Number of Attributes}: 6
  \item[] \bb{Attribute Information}:
  \item[]
  \begin{tabular}{l c c c }
    \bb{Description} & \bb{Values}            & \bb{Type} \\
    buying           & v-high, high, med, low & Discrete  \\
    maint            & v-high, high, med, low & Discrete  \\
    doors            & 2, 3, 4, 5-more        & Discrete  \\
    persons          & 2, 4, more             & Discrete  \\
    lug\_boot        & small, med, big        & Discrete  \\
    safety           & low, med, high         & Discrete
  \end{tabular}
  \item[] \bb{Class Target}: Car Quality
  \item[] \bb{Class Information}:
  \item[]
  \begin{tabular}{l c c c c }
    \bb{Description} & \bb{Value} & \bb{Count} & \bb{Percentage} \\
    Unacceptable     & unacc      & 1210       & 70.023\%        \\
    Acceptable       & acc        & 384        & 22.222\%        \\
    Good             & good       & 69         & 3.993\%         \\
    Very Good        & v-good     & 65         & 3.762\%
  \end{tabular}
  \item[] \bb{Missing Attributes}: none
\end{itemize}

\subsection{E. Coli Data}
\begin{itemize}[leftmargin=*]
  \item[] \bb{Title}: Protein Localization Sites
  \item[] \bb{Date}: September 1997
  \item[] \bb{Number of Samples}: 336
  \item[] \bb{Number of Attributes}: 8
  \item[] \bb{Attribute Information}:
  \item[]
  \begin{tabular}{l c c c }
    \bb{Description} & \bb{Values}       & \bb{Type}  \\
    Sequence Name    &  Accession Number & Discrete   \\
    mcg              & 0.00 - 0.89       & Continuous \\
    gvh              & 0.16 - 1.00       & Continuous \\
    lip              & 0.48 - 1.00       & Discrete   \\
    chg              & 0.50 - 1.00       & Discrete   \\
    aac              & 0.00 - 0.88       & Continuous \\
    alm1             & 0.03 - 1.00       & Continuous \\
    alm2             & 0.00 - 0.99       & Continuous
  \end{tabular}
  \item[] \bb{Class Target}:  Localization Site
  \item[] \bb{Class Information}:
  \item[]
  \begin{tabular}{l c c c c }
    \bb{Description}           & \bb{Value} & \bb{Count} & \bb{Percentage} \\
    Cytoplasm                  & cp         & 143        & 42.56\%         \\
    Inner Membrane             & im         & 77         & 22.92\%         \\
    Perisplasm                 & pp         & 52         & 15.48\%         \\
    Inner Membrane Uncleavable & imU        & 35         & 10.42\%         \\
    Outer Membrane             & om         & 20         & 5.95\%          \\
    Outer Membrane             & omL        & 5          & 1.49\%          \\
    Inner Membrane Lipoprotein & imL        & 2          & 0.59\%          \\
    Inner Membrane Cleavable   & imS        & 2          & 0.59\%
  \end{tabular}
  \item[] \bb{Missing Attributes}: none
\end{itemize}

\subsection{Letter Recognition Data}
\begin{itemize}[leftmargin=*]
  \item[] \bb{Title}: Letter Image Recognition Data
  \item[] \bb{Date}: January 1991
  \item[] \bb{Number of Samples}: 20000
  \item[] \bb{Number of Attributes}: 16
  \item[] \bb{Attribute Information}:
  \item[]
  \begin{tabular}{l c c c }
    \bb{Description} & \bb{Values}            & \bb{Type}  \\
    x-box	           & 0 - 9                  & Continuous \\
    y-box	           & 0 - 9                  & Continuous \\
    width	           & 0 - 9                  & Continuous \\
    high 	           & 0 - 9                  & Continuous \\
    onpix	           & 0 - 9                  & Continuous \\
    x-bar	           & 0 - 9                  & Continuous \\
    y-bar	           & 0 - 9                  & Continuous \\
    x2bar	           & 0 - 9                  & Continuous \\
    y2bar	           & 0 - 9                  & Continuous \\
    xybar	           & 0 - 9                  & Continuous \\
    x2ybr	           & 0 - 9                  & Continuous \\
    xy2br	           & 0 - 9                  & Continuous \\
    x-ege	           & 0 - 9                  & Continuous \\
    xegvy	           & 0 - 9                  & Continuous \\
    y-ege	           & 0 - 9                  & Continuous \\
    yegvx	           & 0 - 9                  & Continuous
  \end{tabular}
  \item[] \bb{Class Target}: Letter
  \item[] \bb{Class Information}:
  \item[]
  \begin{tabular}{l c c c c }
    \bb{Description} & \bb{Value} & \bb{Count} & \bb{Percentage} \\
    Letter A                & A          & 789        & 0.04\% \\
    Letter B                & B          & 766        & 0.04\% \\
    Letter C                & C          & 736        & 0.04\% \\
    Letter D                & D          & 805        & 0.04\% \\
    Letter E                & E          & 768        & 0.04\% \\
    Letter F                & F          & 775        & 0.04\% \\
    Letter G                & G          & 773        & 0.04\% \\
    Letter H                & H          & 734        & 0.04\% \\
    Letter I                & I          & 755        & 0.04\% \\
    Letter J                & J          & 747        & 0.04\% \\
    Letter K                & K          & 739        & 0.04\% \\
    Letter L                & L          & 761        & 0.04\% \\
    Letter M                & M          & 792        & 0.04\% \\
    Letter N                & N          & 783        & 0.04\% \\
    Letter O                & O          & 753        & 0.04\% \\
    Letter P                & P          & 803        & 0.04\% \\
    Letter Q                & Q          & 783        & 0.04\% \\
    Letter R                & R          & 758        & 0.04\% \\
    Letter S                & S          & 748        & 0.04\% \\
    Letter T                & T          & 796        & 0.04\% \\
    Letter U                & U          & 813        & 0.04\% \\
    Letter V                & V          & 764        & 0.04\% \\
    Letter W                & W          & 752        & 0.04\% \\
    Letter X                & X          & 787        & 0.04\% \\
    Letter Y                & Y          & 786        & 0.04\% \\
    Letter Z                & Z          & 734        & 0.04\%
  \end{tabular}
  \item[] \bb{Missing Attributes}: none
\end{itemize}

\subsection{Mushroom Data}
\begin{itemize}[leftmargin=*]
  \item[] \bb{Title}: Mushroom Database
  \item[] \bb{Date}: 27 April 1987
  \item[] \bb{Number of Samples}: 8124
  \item[] \bb{Number of Attributes}: 22
  \item[] \bb{Attribute Information}:
  \item[]
  \begin{tabular}{l c c c }
    \bb{Description}         & \bb{Values}             & \bb{Type} \\
    cap-shape                & b,c,x,f,k,s             & Discrete  \\
    cap-surface              & f,g,y,s                 & Discrete  \\
    cap-color                & n,b,c,g,r,p,u,e,w,y     & Discrete  \\
    bruises?                 & t,f                     & Discrete  \\
    odor                     & a,l,c,y,f,m,n,p,s       & Discrete  \\
    gill-attachment          & a,d,f,n                 & Discrete  \\
    gill-spacing             & c,w,d                   & Discrete  \\
    gill-size                & b,n                     & Discrete  \\
    gill-color               & k,n,b,h,g,r,o,p,u,e,w,y & Discrete  \\
    stalk-shape              & e,t                     & Discrete  \\
    stalk-root               & b,c,u,e,z,r,?           & Discrete  \\
    stalk-surface-above-ring & f,y,k,s                 & Discrete  \\
    stalk-surface-below-ring & f,y,k,s                 & Discrete  \\
    stalk-color-above-ring   & n,b,c,g,o,p,e,w,y       & Discrete  \\
    stalk-color-below-ring   & n,b,c,g,o,p,e,w,y       & Discrete  \\
    veil-type                & p,u                     & Discrete  \\
    veil-color               & n,o,w,y                 & Discrete  \\
    ring-number              & n,o,t                   & Discrete  \\
    ring-type                & c,e,f,l,n,p,s,z         & Discrete  \\
    spore-print-color        & k,n,b,h,r,o,u,w,y       & Discrete  \\
    population               & a,c,n,s,v,y             & Discrete  \\
    habitat                  & g,l,m,p,u,w,d           & Discrete
  \end{tabular}
  \item[] \bb{Class Target}: Edibility
  \item[] \bb{Class Information}:
  \item[]
  \begin{tabular}{l c c c c }
    \bb{Description} & \bb{Value} & \bb{Count} & \bb{Percentage} \\
    Edible           & e          & 4208       & 51.80\%        \\
    Poisonous        & p          & 3916       & 48.20\%
  \end{tabular}
  \item[] \bb{Missing Attributes}: 2480 (all for attribute \#11)
\end{itemize}

\section{Technical Details}
Missing attribute values within the breast cancer dataset were replaced with the mode of the samples of the same class. For the mushroom dataset, as the amount of missing values for attribute 11 was very large, the missing values themselves were simply treated as another category. No discretization occurred as all base algorithms are able to handle continuous data. For all datasets, any sort of identification number or code attributes were removed. Within the E. Coli dataset, the lip and chg categorical attributes were also removed as they provided minimal information. Furthermore, any string values were converted to integers. For example, the ``v-high'', ``high'', ``med'' and ``low'' attribute values would have been converted to 0, 1, 2, 3 respectively.

\subsection{ID3}
\begin{tabular}{ |c|c|c| }
  Dataset            & Max Level & Minimum \# of Samples \\
  Breast Cancer      & None      & SOMETHING            \\
  Car Cancer         & None      & SOMETHING            \\
  Letter Recognition & None      & SOMETHING            \\
  E. Coli            & None      & SOMETHING            \\
  Mushroom           & None      & SOMETHING
\end{tabular}

\subsection{Naïve Bayes}
gaussian

\subsection{Adaboost on Naïve Bayes}
number of learners

\subsection{Adaboost on ID3}
number of learenrs

\subsection{Random Forest}
number of trees

\subsection{K-Nearest Neighbors}
k
distance

\section{Project Design}
A Java machine learning framework was first created to ease the implementation of the six learning algorithms. This framework uses JUnit for unit tests and slfj4 to control logging. Both work independently of the machine learning algorithms. The package hiarchy was modeled off ...

\begin{enumerate}[leftmargin=*]
  \item[] \bb{Algorithm}:
  \item[] \bb{Dataset}:
  \item[] \bb{Model}:
  \item[] \bb{KFold}: Implements the KFold cross validation algorithm.
  \item[] \bb{Report}: The object returned from a k-fold cross validation test....
  \item[] \bb{Util}: Utility functions used through the machine learning project.
\end{enumerate}

\subsection{dt}
This package contains all of the objects associated with decision trees.

\begin{enumerate}[leftmargin=*]
  \item[] \bb{Children}:
  \item[] \bb{ID3}:
  \item[] \bb{ID3Model}:
  \item[] \bb{Node}:
\end{enumerate}

\subsection{ensemble}
This package contains all of the objects associated with enseble learners.

\begin{enumerate}[leftmargin=*]
  \item[] \bb{Adaboost}:
  \item[] \bb{AdaboostModel}:
  \item[] \bb{Multi-Adaboost}:
  \item[] \bb{Multi-AdaboostModel}:
  \item[] \bb{RandomForest}:
  \item[] \bb{RandomForstModel}:
\end{enumerate}

\subsection{exceptions}
This package contains the common exceptions which occur the learning process.

\begin{enumerate}[leftmargin=*]
  \item[] \bb{AttributeException}:
  \item[] \bb{DataException}:
  \item[] \bb{FileException}:
  \item[] \bb{PredictionException}:
\end{enumerate}

\subsection{math}
Data was particularly difficult to handle. Whereas Java is a statically typed language, difficulties arose in handling both floating point and integer values. To manage this, a math package was created which implements useful data structures such as vectors and matrices. These object were used to abstract away from the underlying data structures. Furthermore, the `Vector` object provided many useful methods for vector arithmetic such as element-wise multiplication, division and exponents. These methods also support broadcasting, where a vector can be multiplied by a single number. These objects are used throughout the library in place of primitive arrays and ArrayLists and have were essential to the code quality. Often, methods it was necessary to have two associated objects held in one data structure. For example, there exists a method within the DataSet object which splits itself into two parts based on a pivot point. This method would preferably return a pair of datasets; however, many common data structures such as arrays, lists and maps seemed did not seem to adequate. To accomplish this task, a simple Tuple data structure was created. Several other useful objects were implemented within the math package such as distance function objects, distribution objects and an object containing useful static methods. The purpose of these objects will be elaborated within Naïve Bayes package design section. The utility object implemented useful functions not available within the Java math package.

\begin{enumerate}[leftmargin=*]
  \item[] \bb{MathException}:
  \item[] \bb{Matrix}:
  \item[] \bb{Tuple}:
  \item[] \bb{Util}:
  \item[] \bb{Vector}:
\end{enumerate}

\subsubsection{distance}
This package contains distance functions for the KNN classifier.

\begin{enumerate}[leftmargin=*]
  \item[] \bb{DistanceFunction}:
  \item[] \bb{Euclidian}:
  \item[] \bb{Hamming}:
\end{enumerate}

\subsubsection{distribution}
This package contains distribution implementations for the Naïve Bayes classifier.

\begin{enumerate}[leftmargin=*]
  \item[] \bb{Distribution}:
  \item[] \bb{GaussianDistribution}:
\end{enumerate}

\subsection{nb}
This package contains all of the objects associated with Naïve Bayes.

\begin{enumerate}[leftmargin=*]
  \item[] \bb{Attribute}:
  \item[] \bb{BayesException}:
  \item[] \bb{ClassSummary}:
  \item[] \bb{ContinuousAttribute}:
  \item[] \bb{DiscreteAttribute}:
  \item[] \bb{NaïveBayes}:
  \item[] \bb{NaïveBayesModel}:
\end{enumerate}

\subsection{neighbors}
This package contains all of the objects associated with k-nearest neighbors.

\begin{enumerate}[leftmargin=*]
  \item[] \bb{KNN}:
  \item[] \bb{KNNModel}:
  \item[] \bb{KNNException}:
\end{enumerate}

\section{Results}
\begin{table}
  \begin{tabular}{ |c|c|c|c|c|c| }
    \hline
    Breast Cancer           & Car      & Letter   & E. Coli  & Mushroom \\ \hline
    Adaboost on ID3         & 100.00\% & 100.00\% & 100.00\% & 100.00\% \\ \hline
    ID3                     & 100.00\% & 100.00\% & 100.00\% & 100.00\% \\ \hline
    Random Forest           & 100.00\% & 100.00\% & 100.00\% & 100.00\% \\ \hline
    Naïve Bayes             & 100.00\% & 100.00\% & 100.00\% & 100.00\% \\ \hline
    Adaboost on Naïve Bayes & 100.00\% & 100.00\% & 100.00\% & 100.00\% \\ \hline
    K-Nearest Neighbor      & 100.00\% & 100.00\% & 100.00\% & 100.00\% \\ \hline
  \end{tabular}
  \caption{Results of the six implemented algorithms on five UCI Machine Learning repositories. Accuracy calculated using 10 times 5-fold cross validation.}
\end{table}

graph of hyperparameter search?

\bibliographystyle{plain}
\bibliography{references}
\end{document}
